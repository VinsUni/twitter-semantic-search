\documentclass{acm_proc_article-sp}

\begin{document}

\title{Enabling Semantic Searching on Microblogs with Latent Semantic Analysis
\titlenote{This work is done with the help of Prof. Alfio Gliozzo and Dr. Or Biran in partial satisfaction of the requirements of E6998\_9 Spring 2013 course at Columbia University}
}

\numberofauthors{1} 

\author{\alignauthor Wei Wang, Yu Qiao, Qiuzi Shangguan, Ran Yu \\
\affaddr{Department of Computer Science} \\ 
\affaddr{Columbia University}
\affaddr{New York, USA} \\
\email{\{ww2315,yq2145,qs2130,ry2239\}@columbia.edu}
} 
\maketitle

\begin{abstract}
In our project, we implemented a LSA-based semantic search pipeline over tweets. More particularly, we used TREC 2011 microblog track corpus and its evaluation criteria. The results show that our algorithm yield a XXX performance rise compared with baseline using P@30. 

\end{abstract}

\section{Introduction}
Twitter is the most popular microblogging platform nowadays. There are more than 190 million active users on Twitter and they publish more than 65 million tweets every day \footnote{http://techcrunch.com/2010/06/08/twitter-190-million-users/}. Some research papers show that information retrieved from Twitter can be valuable when used in areas such as earthquake warning [1], opinion mining [2] and fresh website discovering [3]. Growing constantly and involving people's daily social activity, Twitter becomes a goldmine for big data and semantic research [4].\\
TREC (Text REtrieval Conference) encourages research in information retrieval from large text collections. The TREC workshop Microblog Track examines search tasks and evaluation methodologies for information seeking behaviors in microblogging environments \footnote{http://trec.nist.gov/pubs/call2012.html}. The 2012 Microblog Track focuses on addressing two search tasks: Real-time Adhoc and Real-time Filtering whereby a user's information need will be represented by a query at a specific time. In our project, we focus on Real-time Adhoc task: in the real-time search task, the user wishes to see the most recent but relevant information to the query. Hence, the system should answer a query by providing a list of relevant tweets ordered from newest to oldest, starting from the time the query was issued. When scoring tweets, systems should favor relevant and highly informative tweets about the query topic. \footnote{https://sites.google.com/site/microblogtrack/2012-guidelines}

\section{Corpus}
How we get the corpus\\
What it contains\\
The size of the corpus\\

\section{Evaluation Criteria and Baseline}
How TREC evaluates the results\\
How we generated baseline (vs official baseline)\\

\section{Preprocessing}
Remove redundant characters(giiirrl)\\
Remove numbers\\
Remove non-english tweets\\
Dump URLs\\

\section{Rank model}
BM25\\
URL-enhanced ranking formula

\section{LSA}
How distributed LSA works\\

\section{Evaluation Results}
results\\

\section{Conclusion}
What has been done 

\section{Future work}
What needs to be done

\section{Acknowledgments}
Show compliments to Alifo and Or\\  

\bibliographystyle{abbrv}
\bibliography{waton13}
\balancecolumns
\end{document}
